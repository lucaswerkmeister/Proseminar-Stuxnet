\documentclass{beamer}
\usepackage[utf8]{inputenc}
\usepackage{xspace}

\usepackage{templates/beamerthemekit}

\title{Stuxnet}
\subtitle{Proseminar: Desaster in der Software-Sicherheit}
\author{Lucas Werkmeister}
\date{9. Dezember 2014}

\begin{document}

\begin{frame}
  \titlepage
\end{frame}

\begin{frame}
  \frametitle{Einführung}
  \begin{itemize}
    \item Was ist Stuxnet? 0.5, 1.x
    \item Kurz: Wie funktioniert Urananreicherung? Insbesondere im Iran?
    \item Stuxnet-Angriffe
    \item Wie wird die Anlage gesteuert? Wie wird diese Steuerung angegriffen?
  \end{itemize}
\end{frame}

\begin{frame}
  \frametitle{IT-Angriffe} % TODO besserer Titel? Schreiben schädlicher PLC-Blöcke ist immer noch IT…
  \begin{itemize}
    \item Verbreitung
      \begin{itemize}
        \item Step7-Projekte
        \item Memory Sticks
        \item Windows-Netzwerke
      \end{itemize}
    \item Updates
      \begin{itemize}
        \item Command and Control
        \item Peer-to-Peer Updates
      \end{itemize}
    \item Vertiefte Betrachtung:
      \begin{itemize}
        \item CVE-2010-2743: Programmierfehler in Windows
        \item CVE-2012-3015: Step7-Feature als Sicherheitslücke
      \end{itemize}
  \end{itemize}
\end{frame}

\begin{frame}
  \frametitle{Vergleich}
  \begin{itemize}
    \item Stuxnet 0.5: Verbreitung ausschließlich über Step7-Infektionen
    \item Stuxnet 1.x: Breite Vielfalt von Verbreitungsmethoden
    \item Einfachere Verbreitung war wohl nicht mehr ausreichend
    \item \emph{Defense in Depth}-Konzept: Angriff erschweren
  \end{itemize}
\end{frame}

\begin{frame}
  \frametitle{Fazit}
  \begin{itemize}
    \item Stuxnet 1.x wurde durch \emph{Defense in Depth} zu höherer Aggressivität gezwungen
    \item Erhöhte Entdeckungsgefahr musste in Kauf genommen werden
    \item Tatsächlich wurde Stuxnet 1.x recht schnell entdeckt
    \item Stuxnet 0.5 wurde erst 2013 als Vorgänger von Stuxnet 1.x erkannt
  \end{itemize}
\end{frame}

\end{document}
