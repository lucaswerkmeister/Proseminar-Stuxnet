\documentclass{beamer}
\usepackage[utf8]{inputenc}
\usepackage{xspace}

\usepackage{templates/beamerthemekit}

\title{Stuxnet}
\subtitle{Proseminar: Desaster in der Software-Sicherheit}
\author{Lucas Werkmeister}
\date{9. Dezember 2014}

\begin{document}

\newcommand{\ufsechs}[0]{$\mathrm{UF_6}$\xspace}

\begin{frame}
  \titlepage
\end{frame}

\begin{frame}
  \frametitle{Einführung}
  \begin{itemize}
    \item Stuxnet: Zwei Versionen
    \item 0.5: 2007 erstmals gesichtet, erst 2013 als Vorgänger von Stuxnet erkannt
    \item 1.x: 2009 erstmals gesichtet, Sommer 2010 Beachtung gefunden
    \item Gleiches Ziel: Angriff von iranischen Urananreicherungs-Anlagen
  \end{itemize}
\end{frame}

\begin{frame}
  \frametitle{Angriff}
  \begin{itemize}
    \item Kurze Einführung in Funktionsweise der Urananreicherung
    \item Besonderheiten der iranischen Urananreicherung
    \item 1.~Angriff: Ventile
    \item 2.~Angriff: Rotoren
  \end{itemize}
\end{frame}

\begin{frame}
  \frametitle{Steuerung}
  \begin{itemize}
    \item Steuerung der Anlage
    \item PLCs
    \item Step7
    \item Step7-Infektion
  \end{itemize}
\end{frame}

\begin{frame}
  \frametitle{Verbreitung}
  \begin{itemize}
    \item Step7-Projekte
    \item Memory Sticks
    \item Windows-Netzwerke
    \item Command and Control
    \item Peer-to-Peer Updates
  \end{itemize}
\end{frame}

\end{document}
