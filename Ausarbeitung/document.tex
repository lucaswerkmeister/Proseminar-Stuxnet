\documentclass{article}
\usepackage[utf8]{inputenc}
\usepackage{hyperref}

\title{Stuxnet}
\author{Lucas Werkmeister}

\begin{document}

\maketitle

TODO

\section{Einführung}

\subsection{Urananreicherung}

Natürlich vorkommendes Uran besteht zu etwa 99\% aus dem Isotop $^{238}\mathrm U$ und zu etwa 1\% aus $^{235}\mathrm U$.
Allerdings ist nur letzteres Isotop zu einer Kernspaltungs-Kettenreaktion fähig;
zur Anwendung des Urans muss sein Anteil also erhöht werden: auf 3-5\% zur Energiegewinnung, für Nuklearwaffen sogar auf über 85\%.
Diese \emph{Anreicherung} des Urans geschieht üblicherweise durch Gaszentrifugen.
Dabei wird das Gas $\mathrm{UH}_6$, Uranhexafluorid, in einer Zentrifuge zur Rotation gebracht.
Fluor kommt als \emph{Reinelement} in der Natur ausschließlich als $^{19}F$ vor;
jegliche Gewichtsvariation unter den $\mathrm{UH}_6$-Molekülen ist also durch das enthaltene Uran verursacht.
In den Zentrifugen sammelt sich also aufgrund der Zentrifugalkraft das $\mathrm{UH}_6$ mit dem erwünschten $^{235}\mathrm U$ innen,
während $\mathrm{UH}_6$ mit $^{238}\mathrm U$ außen entnommen werden kann.
Durch Wiederholung dieses Prozesses in einer Reihe von Zentrifugen wird das Uran angereichert.

\subsection{Steuerung der Anlage}

Die Urananreicherungsanlage wird durch sog. SCADA-Software kontrolliert.
Es ist unbekannt, welche Software genau im Iran zum Einsatz kommt;
nach einer Analyse von Ralph Langner handelt es sich aber um keines der üblichen Programme, % TODO BibTeX etc
sondern um eine eigene Software.
Das Programm wird als amateurhaft eingestuft;
seine Autoren seien wohl mit modernem SCADA-Design wenig vertraut. % TODO ref

Das SCADA-Programm steuert allerdings nicht direkt die Zentrifugen;
es ist hauptsächlich ein Werkzeug zur Überwachung des Betriebs. % TKaC 9, Seite
Die direkte Steuerung der Zentrifugen geschieht stattdessen durch sog. Programmable Logic Controllers (PLCs).
Dabei handelt es sich um einfache Prozessoren, die Signale von Sensoren verarbeiten,
um verschiedene Ausgangssignale zu setzen.
In der Anlage von Natanz stammt dieser Teil des Systems von Siemens;
der S7-315 Controller steuert die Rotoren der Zentrifugen selbst (164 Motoren pro Controller), % TKaC 12
der S7-417 Controller die Ventile, welche den Zu- und Abfluss von $\mathrm{UH}_6$ regeln (984 Zentrifugen pro Controller).
Diese PLCs werden durch die Siemens-Software \emph{Step7} (Eigenschreibweise \emph{STEP7}) programmiert;
indem Stuxnet dieses Programm infizierte, konnte es eigene Codeblöcke auf die PLCs schreiben
und somit den physikalischen Angriff durchführen.

\section{Angriffe}

Stuxnet führte zwei Angriffe auf das Anreicherungssystem durch, mit dem Ziel, den Prozess zu sabotieren.
Der erste Angriff stammt aus Stuxnet 0.5;
er war in Stuxnet 1.x zwar noch enthalten, aber nicht mehr erreichbar (toter Code):
in Stuxnet 1.x kommt ausschließlich der zweite Angriff zum Einsatz.
Da Stuxnet 0.5 noch nicht bekannt war, als Stuxnet 1.x untersucht wurde,
hielt man den ersten Angriff damals für ``Work in Progress''.
Hätten die Stuxnet-Autoren den ersten Angriff ganz aus Stuxnet 1.x entfernt,
so wäre die Verbindung zwischen den beiden Versionen vermutlich nie aufgefallen,
da sie ansonsten nur noch die Verbreitung über Step7 gemein haben.

\subsection{1. Angriff}

Der erste Angriff manipuliert die Ventile an den Zentrifugen so,
dass $\mathrm{UH}_6$ zwar ein-, aber nicht mehr abfließen kann.
Dadurch steigt der Druck in den Zentrifugen.
Um dies vor den Operatoren der Anlage zu verbergen,
werden alte Sensordaten abgespielt, die Stuxnet vor Beginn des Angriffs aufzeichnet.

Würde Stuxnet nun nichts weiter tun, so würden die Zentrifugen durch den steigenden Druck schließlich zerstört werden.
Da aber ein Controller sehr viele Zentrifugen regelt (s.~o.), würde dies für viele Zentrifugen fast zeitgleich geschehen;
dieses abnormale Verhalten würde auffallen und Stuxnet in einer Post-Mortem-Analyse vermutlich enttarnt werden.
Stuxnet bricht deshalb den Angriff nach einer Weile ab.
Dadurch verursacht dieser Angriff eine höhere Belastung der Anlage,
die durch eine höhere Ausfallrate der Zentrifugen langfristig den gleichen Schaden verursacht,
aber weniger auffällig ist und Stuxnet daher nicht enttarnt.


\end{document}
