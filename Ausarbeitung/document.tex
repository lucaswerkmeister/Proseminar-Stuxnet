\documentclass{article}
\usepackage[utf8]{inputenc}
\usepackage{hyperref}

\title{Stuxnet}
\author{Lucas Werkmeister}

\begin{document}

\maketitle

TODO

\section{Einführung}

\subsection{Urananreicherung}

Natürlich vorkommendes Uran besteht zu etwa 99\% aus dem Isotop $^{238}\mathrm U$ und zu etwa 1\% aus $^{235}\mathrm U$.
Allerdings ist nur letzteres Isotop zu einer Kernspaltungs-Kettenreaktion fähig;
zur Anwendung des Urans muss sein Anteil also erhöht werden: auf 3-5\% zur Energiegewinnung, für Nuklearwaffen sogar auf über 85\%.
Diese \emph{Anreicherung} des Urans geschieht üblicherweise durch Gaszentrifugen.
Dabei wird das Gas $\mathrm{UH}_6$, Uranhexafluorid, in einer Zentrifuge zur Rotation gebracht.
Fluor kommt als \emph{Reinelement} in der Natur ausschließlich als $^{19}F$ vor;
jegliche Gewichtsvariation unter den $\mathrm{UH}_6$-Molekülen ist also durch das enthaltene Uran verursacht.
In den Zentrifugen sammelt sich also aufgrund der Zentrifugalkraft das $\mathrm{UH}_6$ mit dem erwünschten $^{235}\mathrm U$ innen,
während $\mathrm{UH}_6$ mit $^{238}\mathrm U$ außen entnommen werden kann.
Durch Wiederholung dieses Prozesses in einer Reihe von Zentrifugen wird das Uran angereichert.

\end{document}
