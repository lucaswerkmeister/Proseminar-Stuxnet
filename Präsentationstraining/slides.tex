\documentclass{beamer}
\usepackage[utf8]{inputenc}
\usepackage{xspace}

\usepackage{templates/beamerthemekit}

\title{Stuxnet}
\subtitle{Proseminar: Desaster in der Software-Sicherheit}
\author{Lucas Werkmeister}
\date{25. November 2014}

\begin{document}

\newcommand{\ufsechs}[0]{$\mathrm{UF_6}$\xspace}

\begin{frame}
  \titlepage
\end{frame}

\begin{frame}
  \frametitle{Allgemeines}
  \begin{itemize}
    \item Computerwurm, zwei Versionen:
      \begin{itemize}
        \item 0.5: 2007-2009
        \item 1.x: 2010
      \end{itemize}
    \item Gezielter Angriff auf iranische Uran-Anreicherungsanlage
    \item Geschätzter Entwicklungsaufwand: ca.~10~Entwickler, 6~Monate + Quality Assurance, Testen, etc.
    \item Testen erfordert Nachbau der Anlage (inkl. echtes Uran)
    \item Folgerung: Staatliche Angreifer (Annahme: USA)
  \end{itemize}
\end{frame}

\begin{frame}
  \frametitle{Urananreicherung}
  \begin{itemize}
    \item Gaszentrifugen
    \item Uranhexafluorid (\ufsechs)
    \item \ufsechs mit schweren Isotopen sammelt sich außen
    \item Parallelschaltung Zentrifugen: Besserer Durchsatz
    \item Reihenschaltung Zentrifugen: Stärkere Anreicherung
    \item Höhere Frequenzen: Stärkere Anreicherung, aber instabiler
  \end{itemize}
\end{frame}

\begin{frame}
  \frametitle{1. Angriff}
  \begin{itemize}
    \item Manipulation der Ventile an den Zentrifugen
    \item Überdruck in den Zentrifugen
    \item Während des Angriffs werden alte Sensordaten abgespielt
    \item Angriff endet \emph{bevor} irreparabler Schaden entsteht
  \end{itemize}
\end{frame}

\begin{frame}
  \frametitle{2. Angriff}
  \begin{itemize}
    \item Manipulation der Rotorgeschwindigkeit
    \item Überhöhte, dann viel zu niedrige Frequenz
    \item Leicht festzustellen (zuhören)
  \end{itemize}
\end{frame}

\begin{frame}
  \frametitle{Steuerung der Zentrifugen}
  \begin{itemize}
    \item Steuerung durch Programmable Logic Controllers (PLCs)
      \begin{itemize}
        \item Programmierbar (Assembler)
        \item S7-417: Ventile und Drucksensoren
        \item S7-315: Motoren
      \end{itemize}
    \item Controller werden durch Step7-Software kontrolliert + programmiert
    \item Betrieb von normalen Windows-Rechnern aus
    \item Aber ohne Verbindung zum Internet, selbst zum Netzwerk – “air gap”
  \end{itemize}
\end{frame}

\begin{frame}
  \frametitle{Infektion und Verbreitung}
  \begin{itemize}
    \item Memory Sticks
      \begin{itemize}
        \item Autorun.inf (1.001)
        \item LNK (1.10x)
        \item Verwendet, um “air gap” zu überbrücken
        \item Von Subunternehmern eingebracht
      \end{itemize}
    \item Netzwerk
      \begin{itemize}
        \item WinCC
        \item Print Spooler
        \item SMB
      \end{itemize}
    \item Step7-Projekte
  \end{itemize}
\end{frame}

\begin{frame}
  \frametitle{Charakterisierung}
  \begin{itemize}
    \item Stuxnet 0.5
      \begin{itemize}
        \item Angriff 1: Hochkompliziert und spezifisch, schwer zu entdecken
        \item Wenig Verbreitungsmöglichkeiten
        \item 2007 gemeldet
        \item 2013 als Vorgänger von Stuxnet erkannt
      \end{itemize}
    \item Stuxnet 1.x
      \begin{itemize}
        \item Angriff 2: Recht spezifisch, aber eigentlich leicht zu entdecken
        \item Vielfältige Verbreitungsmöglichkeiten
        \item 2010 gemeldet, analysiert, bekannt geworden
        \item Entwickler legten wohl weniger Wert auf Tarnung
      \end{itemize}
  \end{itemize}
\end{frame}

\end{document}
