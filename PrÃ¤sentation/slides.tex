\documentclass{beamer}
\usepackage[utf8]{inputenc}
\usepackage{xspace}
\usepackage{hyperref}

\usepackage{templates/beamerthemekit}

\titleimage{AhmadinejadNatanz_cropped_caption}

\title{Stuxnet}
\subtitle{Proseminar: Desaster in der Software-Sicherheit}
\author{Lucas Werkmeister}
\date{Januar / Februar 2015}

\begin{document}

\newwrite\notesfile
\immediate\openout\notesfile=slides.pdfpc
\immediate\write\notesfile{[file]}
\immediate\write\notesfile{slides.pdf}
\immediate\write\notesfile{[duration]}
\immediate\write\notesfile{30}
\immediate\write\notesfile{[notes]}

\makeatletter
% this is pretty bad:
% 1. it writes '\#' to the file – fix in makefile
% 2. umlauts are not supported
% 3. it results in one 'undefined' error per frame
% 4. multiple notes are not supported ('fix' in makefile: replace LF with line feed)
% but it's good enough for now.
\def\beamer@framenotesbegin{%
  \write\notesfile{\#\#\#\thepage}%
  \write\notesfile{\beamer@notes}%
  \let\beamer@notes\undefined%
}
\makeatother

\begin{frame}
  \titlepage
  \note{Begruessung}
\end{frame}

\begin{frame}
  \frametitle{Allgemeines}
  \begin{itemize}
    \item Computerwurm, zwei Versionen:
      \begin{itemize}
        \item 0.5: 2007-2009
        \item 1.x: 2010
      \end{itemize}
    \item Gezielter Angriff auf iranische Uran-Anreicherungsanlage
    \item Geschätzter Entwicklungsaufwand: ca.~10~Entwickler, 6~Monate + Quality Assurance, Testen, etc.
    \item Testen erfordert Nachbau der Anlage (inkl. echtes Uran)
    \item Folgerung: Staatliche Angreifer (Annahme: USA)
  \end{itemize}
  \note{Hinweise (myrtus) erwaehnen, Tasten im Nebel, nicht drauf eingehen}
\end{frame}

\begin{frame}
  \frametitle{Urananreicherung}
  \begin{itemize}
    \item Gaszentrifugen: Gas mit schwererem Uran sammelt sich außen
    \item 3 Anschlüsse:
      \begin{enumerate}
        \item Feed: Eingang
        \item Product: Ausgang, angereichert
        \item Tails: Ausgang, Rest
      \end{enumerate}
    \item Parallelschaltung Zentrifugen: Besserer Durchsatz
    \item Reihenschaltung Zentrifugen: Stärkere Anreicherung
    \item Höhere Frequenzen: Stärkere Anreicherung, aber instabiler
  \end{itemize}
  \note{foo}
\end{frame}

\begin{frame}
  \frametitle{Urananreicherung im Iran}
  \begin{itemize}
    \item IR-1-Zentrifuge – europäisches Modell der 60er/70er
      \begin{itemize}
        \item ``obsolete design that Iran never managed to operate reliably'' (TKaC 5)
        \item Häufige Ausfälle trotz verringertem Betriebsdruck
        \item Dafür: Massenproduktion
      \end{itemize}
  \end{itemize}
  \note{Best case: nur halbe theoretische Ausbeute}
\end{frame}

\begin{frame}
  \frametitle{1. Angriff}
  \begin{itemize}
    \item Manipulation der Ventile an den Zentrifugen
    \item Überdruck in den Zentrifugen
    \item Während des Angriffs werden alte Sensordaten abgespielt
    \item Angriff endet \emph{bevor} irreparabler Schaden entsteht
  \end{itemize}
  \note{Dampfkochtopf-Analogie?}
\end{frame}

\begin{frame}
  \frametitle{2. Angriff}
  \begin{itemize}
    \item Manipulation der Rotorgeschwindigkeit
    \item Überhöhte, dann viel zu niedrige Frequenz
    \item Leicht festzustellen (zuhören)
  \end{itemize}
  \note{Es werden NICHT alte Sensordaten abgespielt!}
\end{frame}



\begin{frame}
  \frametitle{Quellen}
  \fontsize{6pt}{7.2}\selectfont
  \begin{itemize}
    \item Dossier: Symantec Security Response, W32.Stuxnet Dossier, \url{https://www.symantec.com/content/en/us/enterprise/media/security_response/whitepapers/w32_stuxnet_dossier.pdf}
    \item 0.5: Symantec Security Response, Stuxnet 0.5: The Missing Link, \url{https://www.symantec.com/content/en/us/enterprise/media/security_response/whitepapers/stuxnet_0_5_the_missing_link.pdf}
    \item TKaC: Ralph Langner, To Kill a Centrifuge, \url{http://www.langner.com/en/wp-content/uploads/2013/11/To-kill-a-centrifuge.pdf}
    \item Diese Folien sind auch verfügbar unter \url{https://github.com/lucaswerkmeister/Proseminar-Stuxnet}.
  \end{itemize}
  \note{Weg damit}
\end{frame}

\immediate\closeout\notesfile
\end{document}
